\begin{abstract}
The dynamics of non-native speech perception remain poorly understood, especially in accounting for specialized skills/training. One such skill, musical ability, has been shown to positively impact sensitivity to speech sounds, yet how musical ability is operationalized and measured varies from study to study. Individuals’ musical abilities vary in exposure-duration, skill type (e.g., voice, percussion), and skill-level. In the current study, we take an individual differences (n=\livedata{partrem}{kept_participants}) approach to explore sensitivity in non-native speech discrimination of prosodic contrasts. We measure prosodic sensitivity, working memory, and three measures of musical ability: auditory-motor temporal integration \cite{Kachlicka_Saito_Tierney_2019}, auditory discrimination \cite[MET]{Wallentin_Nielsen_Friis-Olivarius_Vuust_Vuust_2010}), and musical sophistication \cite[Goldsmiths-MSI]{Müllensiefen_Gingras_Musil_Stewart_2014}. We measured prosodic sensitivity using three AX discrimination tasks and signal detection measures (d'): Mandarin tone (primarily cued by pitch), Italian and Japanese (non-)geminates (primarily cued by duration). Results suggest music background, discrimination, and auditory-motor temporal integration capture related –yet divergent– aspects of music experience. Additionally, music sub-skills (e.g., pitch perception) have unequal contributions to non-native speech sensitivity across languages' respective linguistic cues (e.g., tone). Findings support models of non-native speech perception, which consider cognitive factors and auditory experience outside of language experience.

\end{abstract}

\noindent\textbf{Index Terms}:  Individual differences, Music, Non-native speech perception, Measuring prosody
\noindent\section{Introduction}