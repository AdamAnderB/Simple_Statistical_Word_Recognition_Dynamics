
\section{Results}
Results from the Italian linear model found significant effects for: GoldSmiths emotions ($\beta$ = \livedata{modelsouts}{Italian.gs_emotions.estimate}, \text{SE} = \livedata{modelsouts}{Italian.gs_emotions.std.error}, z = \livedata{modelsouts}{Italian.gs_emotions.statistic}, p \textless{} \livedata{modelsouts}{Italian.gs_emotions.p.value}), indicating that higher musical emotional ratings predict lower sensitivity to Italian geminate contrasts; Working memory ($\beta$ = \livedata{modelsouts}{Italian.mean_working_memory.estimate}, \text{SE} = \livedata{modelsouts}{Italian.mean_working_memory.std.error}, z = \livedata{modelsouts}{Italian.mean_working_memory.statistic}, p \textless{} \livedata{modelsouts}{Italian.mean_working_memory.p.value}), indicating that higher working memory capacity predicts better ability to attend to sensitivity needed for Italian geminate contrasts.

Results from the Japanese linear model found significant effects for: GoldSmiths active engagement ($\beta$ = \livedata{modelsouts}{Japanese.gs_activeeng.estimate}, \text{SE} = \livedata{modelsouts}{Japanese.gs_activeeng.std.error}, z = \livedata{modelsouts}{Japanese.gs_activeeng.statistic}, p \textless{} \livedata{modelsouts}{Japanese.gs_activeeng.p.value}), indicating that higher amounts of musical engagement predict lower sensitivity to Japanese geminate contrasts; Working memory ($\beta$ = \livedata{modelsouts}{Japanese.mean_working_memory.estimate}, \text{SE} = \livedata{modelsouts}{Japanese.mean_working_memory.std.error}, z = \livedata{modelsouts}{Japanese.mean_working_memory.statistic}, p \textless{} \livedata{modelsouts}{Japanese.mean_working_memory.p.value}), indicating that higher working memory capacity predicts higher sensitivity to Japanese geminate contrasts;
Auditory-motor Rhythm ($\beta$ = \livedata{modelsouts}{Japanese.beat_score.estimate}, \text{SE} = \livedata{modelsouts}{Japanese.beat_score.std.error}, z = \livedata{modelsouts}{Japanese.beat_score.statistic}, p \textless{} \livedata{modelsouts}{Japanese.beat_score.p.value}), indicating that higher productive rhythmic ability predicts greater sensitivity to Japanese geminate contrasts.

Results from the Mandarin linear model found significant effects for: GoldSmiths emotions ($\beta$ = \livedata{modelsouts}{Mandarin.gs_emotions.estimate}, \text{SE} = \livedata{modelsouts}{Mandarin.gs_emotions.std.error}, z = \livedata{modelsouts}{Mandarin.gs_emotions.statistic}, p \textless{} \livedata{modelsouts}{Mandarin.gs_emotions.p.value}), indicating that higher musical emotional ratings predict lower sensitivity to Mandarin tone; Musical Ear Test melody ($\beta$ = \livedata{modelsouts}{Mandarin.MET_dprime_melody.estimate}, \text{SE} = \livedata{modelsouts}{Mandarin.MET_dprime_melody.std.error}, z = \livedata{modelsouts}{Mandarin.MET_dprime_melody.statistic}, p \textless{} \livedata{modelsouts}{Mandarin.MET_dprime_melody.p.value}), indicating that higher perceptual sensitivity to musical pitch predicts higher sensitivity to Mandarin tone. 

Figure \ref{fig:model} plots the maximal model terms on the y-axis and coefficient estimates on the x-axis. The visualized estimate (point) and confidence interval (error bar) for each language indicates the results from the parsimonious model. In other words, if a term does not have a present coefficient it means that it was not in the parsimonious model of that language.

\begin{figure}[t]
  \centering
  \includegraphics[width=\linewidth]{SP_24_visuals/Japanese,Italian,_Mandarin_max_models_structure:_parsimonious_effects.pdf}
  \caption{Each language's parsimonious model output}
  \label{fig:model}
\end{figure}
