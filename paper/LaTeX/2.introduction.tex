

Non-native speech perception is complex, with many speech contrasts causing difficulties for beginner L2 learners (e.g., \cite{Flege_95,Flege_2021}). For example, Italian and Japanese learners often struggle to differentiate between geminate and non-geminate contrasts (e.g., Japanese non-geminate /kate/ \textit{win}, geminate /kat:e/ \textit{buying}; Italian non-geminate /\textipa{Eko}/ \textit{echo}, geminate /\textipa{Ek\textlengthmark o}/ \textit{here}) \cite{Tsukada_Cox_Hajek_Hirata_2017}. Similarly, Mandarin learners struggle to distinguish between the four tones of Mandarin (e.g., /yu/ means \emph{fish} with a rising tone, \emph{jade} with a falling tone)  \cite{Pelzl_2021}. Yet, it has long been understood that some individuals come to the task of L2 learning with an advantage, evidenced by consistent variation in proficiencies across language sub-skills, while holding study duration and input constant \cite{Zheng_2021}. 

One of the most studied areas of individual differences in non-native speech perception concerns L1 transfer effects. The Perceptual Assimilation Model and Speech Learning Model both successfully demonstrate how a learner’s L1 influences non-native speech perception and learning \cite{Flege_95,Best_1995}. A second but related line of research suggests that individual cues used to discriminate non-native speech contrasts have a major impact on a learner's ability to acquire speech sounds, particularly for prosodic cues found in geminate and tonal contrasts. For example, \cite{Francis_2008} suggests that tone experience in the L1 improves L2 tone discrimination.
Similarly, Japanese speakers start off with 95\% accuracy in Italian geminate discrimination compared to native English speakers with 80\% \cite{Tsukada_Cox_Hajek_Hirata_2017}. Further, \cite{Pajak_2014} shows that both duration and sibilant sensitivity transfer in a gradient manner as a function of how much the respective cue is in the L1, regardless of where the cue occurs in the language. 

More recently, this same line of individual differences research has looked beyond cue similarities in L1 and L2 by examining the role of music training on L2 speech perception. Linguistic and musical cues often share the same acoustic correlates, e.g., F0 and duration. For example, \cite{Wiener_Bradley_2020} demonstrates how short term focused musical pitch training is as beneficial for Mandarin tone discrimination as classroom learning. Beyond this, general musical skills/aptitude and even general cognitive abilities (e.g., \cite{Zheng_2021}) have both been linked to and disputed as successful predictors of improved non-native speech perception. For example, \cite{Zheng_2021} suggests that music aptitude is only a weak predictor of variation in non-native speech perception, whereas general cognitive abilities are more reliable. This is in contrast to \cite{Slevc_2006}, which finds that musical ability is strongly tied to speech-level language abilities. 

These seemingly contradictory results suggest a need for a more nuanced approach to understanding the relationship between individual differences across music subskills and non-native speech discrimination abilities. In the current study, we investigate how different aspects of musical experience predict non-native speech perception by operationalizing music experience in three ways: productive measures (auditory-motor temporal integration) \cite{Kachlicka_Saito_Tierney_2019}, perceptual measures (auditory discrimination) \cite[MET]{Wallentin_Nielsen_Friis-Olivarius_Vuust_Vuust_2010}), and musical sophistication \cite[Goldsmiths-MSI]{Müllensiefen_Gingras_Musil_Stewart_2014}. Beyond music, we use a digit span task as a measure of working memory and measure linguistic sensitivity through three AX discrimination tasks: Italian and Japanese geminate contrasts and Mandarin tone contrasts. Italian, Japanese, and Mandarin were chosen because of their straightforward links from acoustic dimensions: geminate contrast $\rightarrow$ duration;  tone contrast $\rightarrow$ F0. That is, Mandarin sensitivity should be predicted by pitch related skill tasks and Japanese and Italian sensitivity should be predicted by rhythmic skill tasks.

The following three research questions were formulated: 1) To what extent do productive and perceptive measures of musical skill predict non-native speech sensitivity in Italian, Japanese, and Mandarin? 2) To what extent does self-reported music sophistication predict non-native speech sensitivity in Italian, Japanese, and Mandarin? 3) To what extent does working memory predict non-native speech sensitivity in Italian, Japanese, and Mandarin?