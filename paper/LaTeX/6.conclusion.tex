\section{Conclusion}

More broadly, these simulations show that a minimal continuous cue–integration framework can capture core empirical patterns of incremental spoken-word recognition across English, Mandarin, and Italian. Representing segmental, tonal, and stress cues as time-dependent trajectories allows the model to reproduce cohort and rhyme asymmetries, tone-similarity effects, and stress-driven competition without changing the underlying architecture. These differences arise naturally from the temporal structure of the cues and small shifts in cue weighting or representational precision. Although intentionally simple, the framework provides a transparent link between acoustic structure and moment-to-moment lexical activation and offers a flexible basis for future work on prediction, adaptation, and L2 speech learning—highlighting the value of treating prosody as continuous evidence rather than discrete features.