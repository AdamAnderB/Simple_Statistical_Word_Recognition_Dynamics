
\section{Discussion}

The popular conjecture that musical ability is associated with improved language abilities is not a myth \cite{Slevc_Miyake_2006b}, however, the relationship is not as straightforward as the literature may suggest \cite{neves_et_al_2022_music}. With respect to our first research question, we found that musical pitch perceptual abilities do indeed account for the most variance in terms of Mandarin tone sensitivity as previously reported in the literature \cite{zheng_et_al_2018_pitch}. Similarly, the Japanese model's results suggest that rhythm ability is indeed tied to Japanese geminate perception in line with previous theoretical claims \cite{lofqvist_2017_effect}. Interestingly, motor-auditory integration ability rather than perceptual ability is a superior predictor for sensitivity in Japanese geminate contrasts, while perceptual sensitivity to pitch better accounts for variance in Mandarin tone sensitivity, suggesting that different modalities of musical skill play differential roles across language cues. To the best of our knowledge, this is a new finding not reported in the previous music-to-language transfer literature. This is not to say, that the opposite modalities do not predict linguistic sensitivity, but rather that more variance is accounted for when using pitch-perception to predict Mandarin sensitivity and rhythm-production to predict Japanese sensitivity. Surprisingly, this effect does not carry over to the Italian task. It could be that the difference between Mandarin and Japanese model results have to do with how direct the relationship is between the linguistic cue and the musical cue. In the case of Mandarin tones and musical pitch, the same underlying cue of F0 is used to discriminate both whereas the relationship between geminate contrasts and music rhythm is less clear. Although both rhythm and geminate contrasts use duration cues, we measured the ability to attend to the time between beats or the inter-beat interval, while geminate contrasts often rely on the duration of a consonant. In the case of Italian, it could be that no relationship between rhythm and language sensitivity was found because of the changes in vocalic pronunciation between geminate and non-geminate contrasts (co-articulation) \cite{Tsukada_Cox_Hajek_Hirata_2017}. Thus, Italian geminate sensitivity could be better predicted by the ability to discriminate or produce formant differences. 

In terms of research question 2, Table~\ref{tab:comparison} shows that our methods of assessing music ability were reliable. Yet, while specialized skills in music are indeed positively predictive of non-native speech sensitivity, musical self-reported measures have the opposite effect. That is, in all cases throughout the models of each language, higher Goldsmiths music sophistication scores predicted \emph{lower} non-native speech sensitivity. As seen in Figure \ref{fig:centered_data} and corroborated by a post-hoc correlation matrix (see additional OSF materials), Goldsmiths' scores are highly correlated. This unexpected finding could be due to our specific population of university students, the propensity to overestimate ones music ability, or a more general finding that self-reported music ability is not directly associated with any foreign-language ability \cite{larrouy_maestri_et_al_2023_selfevaluation,correia_et_al_2023_selfawareness,schellenberg_et_al_2023_musical}. We tentatively conclude that higher self-reported music ability predicts lower language sensitivity, though further research is needed to clarify this finding. 

Lastly, with respect to research question 3, both the Japanese and Italian models suggest that working memory is a crucial predictor for sensitivity to geminate contrasts. This finding corroborates recent work that demonstrated how working memory can modulate non-native speech category learning success \cite{mchaney_et_al_2021_workingmemory}. Oddly, however, the Mandarin model did not find an effect of working memory. This could be because Mandarin words in our task are single-syllable words based on one syllable, and Japanese and Italian words are varied multi-syllable words. \cite{mchaney_et_al_2021_workingmemory} used five different Mandarin syllables and found an effect of working memory; e.g., increased stimuli variation may also increase cognitive demands. This result could also be due to the dimensionality of tone and geminate contrasts. Unlike tone, which has continuous cues for discrimination throughout a word, geminate contrasts happen only in limited segments of Japanese and Italian words. This could mean that partial information in the Mandarin task is sufficient, but higher cognitive demand is necessary for the Japanese and Italian tasks. Regardless, we conclude that domain general cognitive abilities like working memory play a crucial role in speech perception, particularly in the discrimination of non-native prosody, e.g., \cite{Kachlicka_Saito_Tierney_2019}.
