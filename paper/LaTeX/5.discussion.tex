\section{Discussion}

Together, these simulations show how segmental cues (English), prosodic cues (Mandarin), and combined segmental–prosodic cues (Italian) can be integrated within a single lightweight incremental architecture without discretizing prosody. This same framework also reproduces the core qualitative signatures of spoken-word recognition across the three languages. By treating segmental and prosodic cues as time-dependent trajectories rather than static or discretized features, the model generates activation profiles that mirror well-documented patterns: early cohort and reduced rhyme competition in English, contour-driven divergence among Mandarin tones, and stress-position effects in Italian that emerge only once duration and F0 cues accumulate. These results suggest that many cross-linguistic differences in time course arise from the temporal structure of the cues themselves, not from differences in architectural design.

A consistent outcome across simulations is the central role of cue weighting. In English, strong onset and vowel weighting produced robust cohort dominance and diminished rhyme effects. In Mandarin, high F0 weighting yielded early separation of T1 and T4 and extended coactivation of T2 and T3, while reduced weighting delayed divergence in a manner reminiscent of L2 profiles. In Italian, heavier segmental weighting accelerated the decline of non-stress competitors, yet the key pattern—late divergence at the stressed syllable and sustained activation of the stress-matched item—remained. These results show that simple adjustments to cue attention can account for meaningful variation across languages and listener types.

More broadly, the simulations underscore the value of modeling speech in continuous time. Prosodic cues unfold gradually, and their diagnostic value and ecological validity is lost when forced into segment-aligned units. The present framework offers a transparent alternative: segmental and prosodic cues contribute evidence as they occur, and small changes in cue weighting or representational precision yield realistic differences in processing.